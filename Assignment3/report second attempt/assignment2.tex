%!TEX root = report.tex

\begin{table}[]
\centering
\begin{tabular}{l|l}
Pause     & 1000 \\
Gain      & 1.5    \\
Threshold & 27  
\end{tabular}
\caption{The ideal settings for our scores}
\label{best_settings}
\end{table}

The sweep has been done for the variables \t{pause}(low 600, high 1000), which determines the milliseconds of silence the recognizer should take between sentences, \t{gain}(low 0.5, high 1.5), which controls how much of the background noise gets picked up, and \t{threshold}(low 27, high 33) which determines at which volume sentences start being recognized. We found that the settings seen in Table \ref{best_settings} gave the highest percentages.

\begin{table}[h!]
\centering
\begin{tabular}{l|lll}
Setting          & Sebastiaan & Niels  & Average \\ \hline
default          & 0.0858     & 0.4888 & 0.3261  \\ 
gain high(1.5)   & 0.0834     & 0.486  & 0.3241  \\ 
gain low(0.5)    & 0.084      & 0.4935 & 0.3286  \\ 
pause high(1000) & 0.1066     & 0.5061 & 0.3457  \\ 
pause low(600)   & 0.0575     & 0.4721 & 0.3058  \\ 
high thresh(33)  & 0.0012     & 0.0072 & 0.0049  \\ 
low thresh(27)   & 0.4377     & 0.6258 & 0.549   \\ \hline
Average          & 0.1223     & 0.4399 & 0.3120  \\ 
\end{tabular}
\caption{Results per setting and person.}
\label{results}
\end{table}

The complete results can be seen in Table \ref{results}, where, due to a large discrepency in the results, we showed the differents between Sebastiaan and Niels. As can be seen, the average of Niels is about 4 times higher than Sebastiaan, showing that he was able to produce sentences at a more consistent pace than Sebastiaan.

For the variation of the variables, one can see that gain has almost no influence on the results, accounting for about $0.2\%$ differences in percentage per half point. Unfortunately it is unknown whether our setting for the gain was already over its peak, or if by improving it, we could have obtained a better score.

The pause also had a small influence, accounting for a few percentages difference.

The largest difference was seen when varying the threshold, a high threshold decreased the results by a lot, giving us results of less than $1\%$. A low threshold however, increased our percentages significantly. It is interesting to notice that the differences are a lot larger for Sebastiaan than for Niels. 

For the settings it would be best to use those mentioned in Table \ref{best_settings}, seeing as they individually gave the best performance when altered.

\begin{table}[]
\centering
\begin{tabular}{l|lll}
Setting          & Sebastiaan & Niels  & Average \\ \hline
No grammar       & 0.2313     & 0.265  & 0.2516  \\ 
\end{tabular}
\caption{Results per person when there is no grammar.}
\label{no_grammar}
\end{table}

The scores in Table \ref{results} are calculated using a given grammar file, so the recognizer can only recognize sentences that match this grammar. We also ran the recognizer without a grammar. This drastically increases the amount of possible sentences, which had its effect on the running time of the recognition process. We had to run the \t{decode_all_mixes.py} file overnight to be able to get results without grammar (and the default settings). As can be seen, the results differ a lot. Taking our best default score from Table \ref{results}, which is those of Niels, we see that the result scores are reduced by $22.38\%$ when compared. 

Interesting is that Sebastiaan his scores improves by quite a bit, we believe this is mostly due to the quality of his recordings. So using a grammar file clearly has a positive impact on the recognition, improving the scores by a lot, leading up to results of over $60\%$.
