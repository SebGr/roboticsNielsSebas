%!TEX root = report.tex
We added the file with our adjustments to the file \t{alice_description/complete_model.urdf.xarco}, see \cref{lst:2:adding}.
%
\lstinputlisting[
	caption={Adding the file \t{assignment.urdf.xacro.xml} to \t{complete_model.urdf.xacro.xml}}, 
	label={lst:2:adding}, 
	language=XML,
	firstline=14,
	lastline=15
]{./src/2/complete_model.urdf.xacro.xml}
%
\Cref{lst:2:barLink} presents the link associated with the bar, \t{top_middle_bar} that is placed on top of the \t{middle_bar}. We have assumed that other than its length the bar should have the same dimensions as the \t{middle_bar}, consequently its mass is $\rfrac{0.3}{0.68}$ of the mass of the \t{middle_bar}. The inertia of this bar is the same as that of the \t{middle_bar}. The \t{scale} property of the \t{mesh} element indicates the size the size of the bar. 
%
\lstinputlisting[
	caption={The link of the bar placed on top of the \t{middle_bar}.}, 
	label={lst:2:barLink}, 
	language=XML,
	firstline=14,
	lastline=33
]{./src/2/assignment.urdf.xacro.xml}
%
The \t{top_middle_bar} is connected with the \t{middle_bar} with the joint \t{middle_bar_to_top_middle_bar}. This joint is fixed, since the these two links cannot move relative to each other. We have given the joint an offset of $\rfrac{0.68}{2} + \rfrac{0.3}{2} + 0.02$. This ensures that the \t{top_middle_bar} is placed on top of the \t{middle_bar} and is exactly \SI{30}{\centi\meter} above the top of the \t{middle_bar}.
%
\lstinputlisting[
	caption={The definition of the joint of the bar in \t{assignment.urdf.xacro.xml}.},
	label={lst:2:barJoint}, 
	language=XML,
	firstline=36,
	lastline=40
]{./src/2/assignment.urdf.xacro.xml}
%
\lstinputlisting[
	caption={The definition of the joint \t{middle_xtion_top_middle_bar} in \t{assignment.urdf.xacro.xml}.},
	label={lst:2:xion}, 
	language=XML,
	firstline=118,
	lastline=122
]{./src/2/assignment.urdf.xacro.xml}
%
The complete code can be found in \cref{sec:a:ass1}. \Cref{lst:2:xion} presents the definition of the joint \t{middle_xtion_top_middle_bar}. The child link \t{middle_xtion_link} is not shown here because it is a copy from the \t{front_xtion_link}. In order to get the correct angle, we tilted the camera $3/4 * \pi$ so that it was facing downwards.\\

Because the camera had to be on top of the bar, we calculated the distance for the joint, that it had to be upwards on the z-axis. Using basic geometry we calculated that the camera would have to be placed $15+2.121$ upwards on the z-axis in order to put it on the top of the bar, with the lowest point still touching the bar.